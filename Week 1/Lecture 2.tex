\documentclass{article}

\usepackage{mathtools}
\usepackage{amssymb}
\usepackage{amsthm}
\usepackage{enumerate}

\begin{document}

\title{Lecture 2 Notes}
\author{David Gieseker}
\maketitle

\newtheorem*{theorem}{Theorem}
\newtheorem*{corollary}{Corollary}
\newtheorem*{lemma}{Lemma}
\newtheorem*{definition}{Definition}
\newtheorem*{example}{Example}

\section{Euclid's Knowledge}
Let's first review the division algorithm. 
\begin{theorem}
Let $a,b \in\mathbb{Z}$ and $b>0$. Then, $\exists! q,r\in\mathbb{Z}$ such that $a=qb+r$ where $0 \leq r<b$. 
\end{theorem}
This is the same as what we learned from elementary school. We know that $q$ is the quotient and $r$ is the remainder -- and both are unique.
%\begin{example}
%$7=2\cdot3+1, -7=-3\cdot3+2$
%\end{example}
\begin{proof}
(existence) Let $S=\{n\in\mathbb{Z}\mid\exists x\in\mathbb{Z}, n=a-bx\geq0\}$. Suppose that $S\neq\emptyset$ (the elements are potential $q$'s). Then, we can consider 3 different cases: 
\begin{enumerate}
	\item If $a\geq0$, we take $x=0$ and $a\in S$. 
	\item If $a<0$, we take $x=a$. Then, $n=a-ab=a(1-b)$. But since $b>0$ and $b\in\mathbb{Z}$, $n=a(1-b)\geq0$. Therefore, $a\in S$. 
\end{enumerate}
And by the Well-Ordering Property, there is a least element $r$ of $S$. If we denote its corresponding $x$ to be $q$, then $r=a-bq\geq0 \implies a=qb+r$. \\
Next, we show that $0\leq r < b$. We showed $r\geq0$ previously. Then assume that $r\geq b$, and thus $0\leq r-b = a-b(q+1)\in S$. However, $r$ is assumed to be a least element of $S$, so $r\leq r-b$ and $b\leq0$. This is a contradiction, so $r < b$.
\end{proof}
\begin{proof}
(uniqueness) Assume that quotients and remainders are not unique. Then $a=qb+r=q'b+r'$, where $0\leq r<b, 0\leq r'<b$. If the remainder is not unique, then we can take $r>r'$ without loss of generality. Thus, we have $0<r-r'\leq r<b$. And since $0\neq q-q'\in\mathbb{Z}$, $|q-q'|\geq1$. Then from the original equation, we have $r-r'=(q'-q)b\geq b$. This is a contradiction to the prior inequality. Thus, $r=r'$ must hold, and $q=q'$ follows as $b>0$. 
\end{proof}
\begin{definition}
Let $d,m\in\mathbb{Z}$ where $d\neq0$. Then $d$ divides $m$ if $\exists e\in\mathbb{Z}, m=ed$, notated by $d\mid m$.
\end{definition}
\begin{definition}
Let $m,n\in\mathbb{Z}$. Then, $d\in\mathbb{Z}$ is a greatest common divisor, notated by $d=(m,n)$, of $m$ and $n$ if: 
\begin{enumerate}[(i)]
	\item $d>0$
	\item $d\mid m, d\mid n$
	\item if $e\in\mathbb{Z}$ and $e\mid m, e\mid n$, then $e|d$. 
\end{enumerate}
In other words, $d$ is a divisor of $m$ and $n$ that divides any other common divisors. 
\end{definition}
\begin{definition}
$f\in\mathbb{Z}$ is a $\mathbb{Z}$ linear combination of $m,n\in\mathbb{Z}$ if $\exists x,y\in\mathbb{Z}$ such that $f=xm+yn$.
\end{definition}
\begin{theorem}
Let $m,n\in\mathbb{Z}$, and at least one of which is nonzero. Then, $d=(m,n)$ exists, is unique, and is a $\mathbb{Z}$ linear combination of $m$ and $n$. 
\end{theorem}
\begin{proof}
Let's define $S=\{am+bn>0\mid a,b\in\mathbb{Z}\}$. Now, we first show that $d\mid m$. \\
$S$ is nonempty, since we can always take $a=sgn(m),b=0$ and $am+bn\in S$ must hold. Thus, it must have a least element $d$. Then by the division algorithm, $\exists q,r\in\mathbb{Z}$ such that $m=qd+r$ and $0\leq r<d$.
$$r=m-qd=m-(am+bn)q=(1-aq)m-bqn\geq0$$
Therefore, $r$ is also a $\mathbb{Z}$ linear combination of $m$ and $n$. However $r<d$ and $d$ is the least element of $S$, so $r=0$ is the only possibility. Therefore, $m=qd\implies d\mid m$. Similarly, $d\mid n$ can be proven with a similar method. \\
Now, suppose that $e$ is another common divisor of $m$ and $n$. Thus $m=xe$ and $n=ye$, so $d=(ax+by)e$ and it follows that $e|d$. Therefore $d=(m,n)$ by definition, and it is a $\mathbb{Z}$ linear combination of $m$ and $n$ as it is in $S$. 
\end{proof}
\begin{proof}
(uniqueness) Let $d,d'$ be two GCD's of $m$ and $n$. Then, $d|d'$ and $d'|d$ must both hold. This is true only when $d=\pm d'$, and since $d,d'>0$ they must be equal. 
\end{proof}
\end{document}